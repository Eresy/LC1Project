\documentclass[a4paper,oneside,11pt]{article}
\usepackage{graphicx}
\usepackage[italian]{babel}
\usepackage{amsmath}
\usepackage{amssymb}
\usepackage{standalone}
\usepackage{pgf}
\usepackage{tikz}
\usetikzlibrary{arrows,automata,positioning}
%use this for direct use of àèìòù
\usepackage[utf8x]{inputenc}
\usepackage[T1]{fontenc}
\usepackage{listings}
\usepackage{fancyvrb}
%
% fancy footnotes
\usepackage[perpage, symbol*, stable, multiple]{footmisc}
\makeatletter
\renewcommand\@makefnmark{\@textsuperscript{\normalfont (\@thefnmark)}}
\makeatother
%
% fancy captions 
\usepackage[format=default, labelsep=period, font=small, labelfont=sc]{caption}
%\captionsetup[table]{position=top}
%
%
% redefines the maketitle
\makeatletter
\renewcommand\and{\\}
\renewcommand\maketitle{% 
\bigskip\bigskip\bigskip\bigskip%
\begin{center}\bfseries\large%
Progetto di Linguaggi e Compilatori 1 -- Parte 2 \\ A.A. 2015/16\\%
\end{center}%
\bigskip%
\begin{center}\bfseries\LARGE \@title  \end{center}%
\bigskip%
\begin{center}\bfseries\large \@author \end{center}%
\bigskip\bigskip}
\makeatother
%
% the paths where looking for figures
%\graphicspath{{./}, {eps/}}
%
%
\begin{document}
\title{Gruppo 14}
\author{Marco Bucchiarone \and Emanuele Tonetti \and Francesco Zilli}
\maketitle
%
\section*{Esercizio 1}
\subparagraph*{(a)}
Dato un testo formattato come 
\begin{center}
cognome  nome/nomi  data(gg/mm/aa)  matricola  altro-testo
\end{center}
con i campi separati da un numero arbitrario di spazi, le espressioni regolari, nella sintassi di flex, componenti l'espressione regolare $e_{in}$ per eseguire la riformattazione del testo sono:
\begin{itemize}
	\item[cognome]		$($\Verb/[a-zA-Z\-\']+/$)_{1}$
	\item[nome/nomi]	$($\Verb/([a-zA-Z\-])+([ ]+[a-zA-Z\-]+)*/$)_{2}$
	\item[gg]		$($\Verb/(0?[1-9])|([12][0-9])|3[01]/$)_{3}$ 
	\item[mm]		$($\Verb/(0?[1-9])|(1[0-2])/$)_{4}$ 
	\item[aa]		$($\Verb/(0?[0-9])|([1-9][0-9])/$)_{5}$ 
	\item[matricola]	$($\Verb/[0-9]+{6}/$)_{6}$ 
	\item[separatore]	$($\Verb!"/"!$)$ 
	\item[spazi]		$($\texttt{[\textvisiblespace]+}$)$
	\item[altro-testo]	$($\Verb/./$)$ 
\end{itemize}
dove, per semplicità di notazione, sono state numerate solo le parentesi contenenti le $regexp$ facenti il match dei campi che si vuole siano presenti nell'espressione $e_{out}$.
Quindi la $regexp$ $e_{in}$ assumerà forma
\begin{center}
	$e_{in} =${cognome}{spazi}{nome/nomi}{spazi}{gg}{separatore}
\end{center}
\par
Volendo $e_{out}$ della forma
\begin{center}
	matricola nome/nomi cognome data(aaaa-mm-gg)
\end{center}
con i campi separati con un tabulatore ed assumendo che, tutte le date successive al 2000, non abbiano singole cifre non precedute da 0, si avrà
\begin{center}
	$e_{out}=$\Verb!\6\t\2\t\1\t20\5-\4-\3!
\end{center}
dove \Verb!\t! indica il carattere di tabulazione.
\par
\includestandalone[width=\textwidth]{dfa}
\section*{Esercizio 2}
\section*{Esercizio 3}
compilato con bison 3.0.4
flex 2.6.0

\end{document}
