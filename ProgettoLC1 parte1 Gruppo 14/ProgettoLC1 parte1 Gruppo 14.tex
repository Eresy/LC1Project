%listfiles%prints on standard output the list of files used by this document
\documentclass[a4paper, oneside, 11pt]{article}
%
% useful packages
\usepackage[dvips]{graphicx}
\usepackage[italian]{babel}
\usepackage{amsmath}
\usepackage{amssymb}
% 
%\begin{figure}
%\centering
%\includegraphics[width=\textwidth]{filename.eps}
%\caption{.}
%\label{labelname}
%\end{figure}
%
% to built tables longer than one page (here needed to print all the data)
\usepackage{longtable}
%
% to print numerical values in a format suitable for scientific papers
\usepackage{sistyle}
%\SIstyle{USA}
%\SI{0.01}{m}
%
% fancy footnotes
\usepackage[perpage, symbol*, stable, multiple]{footmisc}
\makeatletter
\renewcommand\@makefnmark{\@textsuperscript{\normalfont (\@thefnmark)}}
\makeatother
%
% fancy captions 
\usepackage[format=default, labelsep=period, font=small, labelfont=sc]{caption}[2004/07/16]
\captionsetup[table]{position=top}
%
% useful general commands
\newcommand*{\eqname}[0]{Eq.}
\newcommand*{\eqspace}[0]{\:}
\providecommand{\eqref}[1]{(\ref{#1})}
%
% redefines the maketitle
\makeatletter
\renewcommand\and{\\}
\renewcommand\maketitle{% 
\bigskip\bigskip\bigskip\bigskip%
\begin{center}\bfseries\large%
Progetto di Linguaggi e Compilatori 1 -- Parte 1 \\ A.A. 2015/16\\%
\end{center}%
\bigskip%
\begin{center}\bfseries\LARGE \@title  \end{center}%
\bigskip%
\begin{center}\bfseries\large \@author \end{center}%
\bigskip\bigskip}
\makeatother
%
% the paths where looking for figures
\graphicspath{{./}, {eps/}}
%
%
\begin{document}
\title{Gruppo 14}
\author{Marco Bucchiarone \and Emanuele Tonetti \and Francesco Zilli}
\maketitle
%
\section*{Esercizio 1}
La funzione Haskell \texttt{boundedMaximum(n [BST t])} data una lista di BST \textit{t} trova, se esiste  \[max_{n}{t}=max(\forall x \in t \mid  x<n ).\]
La soluzione all'esercizio \`e implementata nel modulo BoundMax (\texttt{BoundMax.hs}) che viene importato nel \texttt{Main}.
Nel \texttt{Main} la funzione viene chiamata fornendole gli argomenti inseriti a tastiera a runtime (usando le funzioni importate tramite \texttt{System.Environment}).
\par
La \texttt{boundedMaximum} viene costruita sulla \texttt{foldl1}: \texttt{MaxNT} individua il massimo tra i minoranti nella lista costruita dalla funzione di visita del BST \texttt{getmin} che sfrutta la struttura dati considerata.
Il modulo \texttt{BoundMax} \`e testabile nell'interprete di GHC tramite i test case forniti nel file \texttt{queryEs1.txt}, il quale comprende anche una rappresentazione grafica di alcuni degli alberi usati per una maggiore leggibilit\`a.
%
\section*{Esercizio 2}
\subparagraph*{Parte A:}
Dalla sintassi concreta per alberi pesati con numero arbitrario di figli fornita \`e stata costruita la sintassi astratta \textit{polimorfa}.
L'implementazione proposta \`e suddivisa tra i seguenti file, di cui si da una breve descrizione:
\begin{description}
	\item[Data.hs] contiene la definizione dei tipi di dato per la sintassi astratta polimorfa  e per i token,
	\item[Lexer.x] produce token per i soli elementi dell'alfabeto usati dalla sintassi concreta comprese le rappresentazioni dei dati di tipo \texttt{Int} e \texttt{Double} ignorando elementi estranei all'alfabeto,
	\item[ParserI.y] produce di alberi pesati di numeri soli \texttt{Int},
	\item[ParserD.y] produce di alberi pesati di numeri \texttt{Double}.
%	\item[Symm.hs] 
%	\item[TestSymm.hs]
\end{description}	
\par
In entrambi i parser \`e fornita, nella sezione di codice opzionale, la funzione \texttt{detWeight} che calcola il peso dei nodi individuali; il peso viene definito ricorsivamente come la distanza massima di una foglia dal nodo considerato, incrementato di 1 (ad ogni foglia viene assegnato peso pari a 0). 
Nel caso del parser per alberi (pesati) in virgola mobile si \`e assunto i \texttt{Double} siano rappresentabili da sequenze di cifre da 0 a 9 senza punto decimale.
Dalla formulazione della consegna si \`e assunto la grammatica \textit{G}, implementata nel parser, non abbia produzioni di tipo \(\epsilon\).
\subparagraph*{Parte B:}
Considerando l'alfabeto \(\Sigma=\{\texttt{[},\texttt{]},\texttt{,},\texttt{a},\texttt{b}\}\), l'insieme \[\mathcal{P}=\{w \in \Sigma^{\ast}\mid w=w^{R}\}\] \`e l'insieme delle stringhe di elementi dell'alfabeto palindrome. 
Siccome il linguaggio \(\mathcal{L}(G)\) generato dalla grammatica ammette come stringhe palindrome solo gli alberi di altezza zero rappresentati dal singolo nodo "\texttt{a}" e "\texttt{b}", l'insieme risultante da \[\mathcal{P}\setminus\mathcal{L}(G)\] corrisponde all'insieme ottenuto facendo %
\[\mathcal{P}\setminus\{x\in\Sigma^{\ast}\mid x=a \vee x=b\}.\]
\par
Dimostrazione:


\newpage
In entrambi i parser \`e implementata, come codice aggiuntivo, la funzione detWeight che determina per ogni singolo nodo il rispettivo peso (sommando 1 al peso massimo di eventuali figli), il quale sarà fornito nel risultato.

   La funzione isSymm utilizza due sotto-funzioni:

- isEq: stabilisce l'equivalenza strutturale di due alberi;

- trasp: la trasposta di un albero in maniera ricorsiva (invertendo i suoi figli);

isSymm si basa sull'idea che una foglia \`e sempre simmetrica a se stessa, mentre un nodo lo \`e solamente se equivale al suo trasposto.


- isEq: stabilisce l'equivalenza strutturale di due alberi;

- trasp: la trasposta di un albero in maniera ricorsiva (invertendo i suoi figli);

isSymm si basa sull'idea che una foglia \`e sempre simmetrica a se stessa, mentre un nodo lo \`e solamente se equivale al suo trasposto.
   La funzione isSymm \`e combinata al parser per interi nel file sorgente TestSymm; quivi viene stampato a video l'albero, il rispettivo albero trasposto ed il risultato della funzione isSymm che indicherà se l'albero \`e effettivamente simmetrico.
  All'eseguibile risultante dalla compilazione \`e possibile fornire in input il file "examples.txt" per effettuare svariati test case.
  (eseguire make demo per veloce riscontro)

  - breve descrizione della grammatica w=wR con esempi

- breve descrizione della grammatica di L(G) con esempi, di cui palindrome solo "a", "b"

- descrizione della sottrazione P - L(G) con dimostrazione

- descrizione dell'insieme risultante con esempi ("[,[" e/o "ab[ba")
  (G)



\end{document}
