\documentclass[a4paper,oneside,11pt]{article}
\usepackage{graphicx}
\usepackage[italian]{babel}
\usepackage{amsmath}
\usepackage{amssymb}
\usepackage{standalone}
%use this for direct use of àèìòù
\usepackage[utf8x]{inputenc}
\usepackage[T1]{fontenc}
\usepackage{listings}
\usepackage{fancyvrb}
%
%let's print BNF production the nice way
\usepackage{syntax}
%syntax specific option
\def\<#1>{\synt{#1}}
\setlength{\grammarindent}{7em} % increase separation between LHS/RHS 
\setlength{\grammarparsep}{15pt plus 1pt minus 1pt} % increase separation between rules
%
\usepackage{hyperref}
%
% fancy footnotes
\usepackage[perpage, symbol*, stable, multiple]{footmisc}
\makeatletter
\renewcommand\@makefnmark{\@textsuperscript{\normalfont (\@thefnmark)}}
\makeatother
%
% fancy captions 
\usepackage[format=default, labelsep=period, font=small, labelfont=sc]{caption}
%\captionsetup[table]{position=top}
%
%
% redefines the maketitle
\makeatletter
\renewcommand\and{\\}
\renewcommand\maketitle{% 
\bigskip\bigskip\bigskip\bigskip%
\begin{center}\bfseries\large%
Progetto di Linguaggi e Compilatori 1 -- Parte 3 \\ A.A. 2015/16\\%
\end{center}%
\bigskip%
\begin{center}\bfseries\LARGE \@title  \end{center}%
\bigskip%
\begin{center}\bfseries\large \@author \end{center}%
\bigskip\bigskip}
\makeatother
%
% the paths where looking for figures
%\graphicspath{{./}, {eps/}}
%
%
\begin{document}
\title{Gruppo 14}
\author{Marco Bucchiarone \and Emanuele Tonetti \and Francesco Zilli}
\maketitle
%
\section*{Assunzioni}
Nel costruire il nostro linguaggio basato su \href{http://chapel.cray.com}{Chapel} ci siamo basati sulle regole grammaticali esposte nella \href{http://chapel.cray.com/docs/latest/_downloads/chapelLanguageSpec.pdf}{Chapel Language Specification}.
Tuttavia alcune regole di produzione risultavano in contrasto con le forme incontrate in altri documenti ufficiali quali il \href{http://chapel.cray.com/docs/latest/_downloads/quickReference.pdf}{Quick Reference} e \href{http://chapel.cray.com/papers/BriefOverviewChapel.pdf}{A Brief Overview of Chapel} il che ci ha portato a fare delle semplificazioni sulle specifiche finali del linguaggio.
Infatti benchè la regola di produzione degli \synt{statements} abbia forma
\begin{grammar}
	   <Statements> ::= Statement | Statement Statements
\end{grammar}
negli esempi di codice di Chapel gli \synt{Statements} usano il \lit{;} come separatore; per questa ragione abbiamo assunto un uso del separatore \lit{;} simile a quello di C. 
\par
Per ragioni di semplificazione si è deciso di ridurre le produzioni di \synt{WhileDo-stmt}, \synt{DoWhile-stmt}, \synt{For-stmt} ed \synt{IF-stmt} dalla grammatica di Chapel alle seguenti regole:
\begin{grammar}
<IF_stmt> ::= "if" <Exp> "then" <Stmt>
	\alt "if" <Exp>  <Block_stmt>
	\alt "if" <Exp>  <Block_stmt> "else" <Stmt>

<While_stmt> ::= "while" <Exp> <Block_stmt>

<Do_stmt> ::= "do" <Block_Stmt> "while" <Exp>

<For_stmt> ::= "for" <Id> "in" <Range_exp> <Block_stmt>
\alt "for" <Range_exp> "do" <Stmt>
\end{grammar}

\section*{Svolgimento}

\end{document}
