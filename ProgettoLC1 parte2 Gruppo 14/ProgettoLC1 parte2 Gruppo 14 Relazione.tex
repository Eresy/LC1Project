\documentclass[a4paper,oneside,11pt]{article}
\usepackage{graphicx}
\usepackage[italian]{babel}
\usepackage{amsmath}
\usepackage{amssymb}
\usepackage{standalone}
\usepackage{pgf}
\usepackage{tikz}
\usetikzlibrary{arrows,automata,positioning}
%use this for direct use of àèìòù
\usepackage[utf8x]{inputenc}
\usepackage[T1]{fontenc}
\usepackage{listings}
\usepackage{fancyvrb}
%
% fancy footnotes
\usepackage[perpage, symbol*, stable, multiple]{footmisc}
\makeatletter
\renewcommand\@makefnmark{\@textsuperscript{\normalfont (\@thefnmark)}}
\makeatother
%
% fancy captions 
\usepackage[format=default, labelsep=period, font=small, labelfont=sc]{caption}
%\captionsetup[table]{position=top}
%
%
% redefines the maketitle
\makeatletter
\renewcommand\and{\\}
\renewcommand\maketitle{% 
\bigskip\bigskip\bigskip\bigskip%
\begin{center}\bfseries\large%
Progetto di Linguaggi e Compilatori 1 -- Parte 2 \\ A.A. 2015/16\\%
\end{center}%
\bigskip%
\begin{center}\bfseries\LARGE \@title  \end{center}%
\bigskip%
\begin{center}\bfseries\large \@author \end{center}%
\bigskip\bigskip}
\makeatother
%
% the paths where looking for figures
%\graphicspath{{./}, {eps/}}
%
%
\begin{document}
\title{Gruppo 14}
\author{Marco Bucchiarone \and Emanuele Tonetti \and Francesco Zilli}
\maketitle
%
\section*{Esercizio 1}
\subsection*{(a)}
Dato un testo formattato come 
\begin{center}
cognome  nome/nomi  data(gg/mm/aa)  matricola  altro-testo
\end{center}
con i campi separati da un numero arbitrario di spazi.
Le espressioni regolari, nella sintassi di flex, componenti l'espressione regolare $e_{in}$ per eseguire la riformattazione del testo sono:
\begin{itemize}
	\item[cognome]		$($\Verb/[a-zA-Z\-\']+/$)_{1}$
	\item[nome/nomi]	$($\Verb/([a-zA-Z\-])+([ ]+[a-zA-Z\-]+)*/$)_{2}$
	\item[gg]		$($\Verb/(0[1-9])|([12][0-9])|3[01]/$)_{3}$ 
	\item[mm]		$($\Verb/(0[1-9])|(1[0-2])/$)_{4}$ 
	\item[aa]		$($\Verb/[0-9]{2}/$)_{5}$ 
	\item[matricola]	$($\Verb/[0-9]{6}/$)_{6}$ 
	\item[separatore]	$($\Verb!"/"!$)$ 
	\item[spazi]		$($\texttt{[\textvisiblespace]+}$)$
	\item[altro-testo]	$($\Verb/./$)$ 
\end{itemize}
dove, per semplicità di notazione, sono state numerate solo le parentesi contenenti le $regexp$ facenti il match dei campi che si vuole siano presenti nell'espressione $e_{out}$.
\newpage
\par
Quindi la $regexp$ $e_{in}$ assumerà forma
\begin{center}
%$e_{in} =$\Verb!{cognome}{spazi}{nome/nomi}{spazi}{gg}{separatore}{mm}{separatore}{aa}\-{spazi}{matricola}{spazi}{altro-testo}!
$e_{in} =$\Verb!{cognome}!\Verb!{spazi}!\Verb!{nome/nomi}!\Verb!{spazi}!
\Verb!{gg}!\Verb!{separatore}!\Verb!{mm}!\Verb!{separatore}!\Verb!{aa}!
\Verb!{spazi}!\Verb!{!\Verb!matricola}!\Verb!{spazi}!\Verb!{altro-testo}!.
\end{center}
\par
Volendo $e_{out}$ della forma
\begin{center}
	matricola nome/nomi cognome data(aaaa-mm-gg)
\end{center}
con i campi separati con un tabulatore ed assumendo che, tutte le date successive al 2000, non abbiano singole cifre non precedute da 0, si avrà
\begin{center}
	$e_{out}=$\Verb!\6\t\2\t\1\t20\5-\4-\3!
\end{center}
dove \Verb!\t! indica il carattere di tabulazione.
\subsection*{(b)}
Preso l'alfabeto $\Sigma$ contenente i caratteri ASCII, il DFA minimo per $e_{in}$ è:
\includestandalone[width=\textwidth]{dfa}\label{dfa}
dove, per chiarezza illustrativa, si sono indicati sugli archi il range di caratteri che, ricevuti dallo stato $q_i$ causano la transizione allo stato $q_{j}$. Per la stessa ragione è stata omessa la rappresentazione dello stato "pozzo" a cui puntano tutti gli stati qualora ricevano un carattere non accettato dalle transizioni esplicitate.
\newpage
\section*{Esercizio 2}
\lstset{tabsize=6}
\begin{lstlisting}
init:
		ENT 2       -- variabile u e variabile temporanea 
			      -- necessaria all'impl. del for
		LDA 0 9     -- carico la variabile temporanea
		MST 1 
		LDA 1 4     -- IMAX
		IND         
		LDA 0 5	    -- j di init
		IND
		CUP 2 min   -- min(imax,j)
		STO         -- temp := min(IMAX, j)
		LDA 0 8	    -- carico u
		LDC int 0 
		STO         -- u := 0
		UJP guard-init  -- while u <= temp do S; u += 1;
body-init:
		MST 1
		LDA 0 8
		IND         -- u
		LDA 0 4
		IND         -- i
		MUL int     -- u * i
		LDA 0 7
		IND          
		CUP 2 p	    -- p(u*i , z)
		LDA 0 6     -- h : history
		IND         -- history := ^harray
		LDA 0 4     -- index i
		IND
		IXA 10      -- h[i][] pointer
		LDA 0 8     -- index u
		IND
		IXA 1       -- h[i][u] pointer
		LDA 1 8	    -- mainharray(offset 8)
		IND
		LDA 0 8     -- index u
		IND
		IXA 10      -- [u] pointer
		LDA 0 5     -- index j
		IND 
		IXA 1       -- [u][j] pointer
		IND       -- mainharray[u,j]
		STO	    
		LDA 0 8     
		LDA 0 8   -- duplico la u
		IND
		LDC int 1
		SUM int
		STO
guard-init:
		LDA 0 8   -- u
		IND
		LDA 0 9   -- temp
		IND
		GTR
		FJP body-init
		RETP
f:
		ENT 0
		...
		RETF
p:
		ENT 0     
		LDA 0 5
		IND       
		MST 1     -- pre-chiamata al primo f
		MST 1     -- pre-chiamata al secondo f
		LDA 0 4
		IND
		FLT       -- f() si aspetta real
		CUP 1 f
		CUP 1 f
		STO       -- y:= f(f(x))
		UJP guard-p
body-p:
		MST 1     -- preparo per chiamata ricorsiva di p(y,y)
		LDA 0 5
		IND
		IND       
		LDA 0 5
		IND       
		CUP 2 p
guard-p:
		LDA 0 5
		IND
		IND
		LDC int 0
		NEQ       
		LDA 0 4
		IND 
		LDA 1 6  
		IND     
		LEQ       
		OR
		FJP body-p
		RETP
min:
		ENT 0
		...
		RETF
f:		          --interna ad alt
		ENT 0
		LDA 0 0	
		LDC real 1
		LDC real 1
		LDA 0 4  
		IND
		DIV real
		SUM real
		STO     
		RETF
alt:
		ENT 0
		LDA 0 4
		IND       -- carico i
		ODD
		FJP else
then:
		LDA 0 0
		MST 1     -- preparo chiamata ricorsiva alt
		LDA 0 4
		IND
		LDC int 1
		SUB int
		MST 0     -- preparo chiamata alla funzione interna f;
		LDA 0 5
		IND
		CUP 1 f
		CUP 2 alt
		STO
		RETF
else:
		LDA 0 0
		MST 1
		LDA 0 4
		IND
		LDC int 1
		SUB int 
		LDA 0 5
		IND     
		CUP 2 alt
		STO
		RETF
main:
		...
		MST 0     
		LDA 0 6	-- target
		IND
		LDA 0 7	-- aim
		CUP 2 p
		MST 0       
		LDC int 20
		LDC int 30
		LDA 0 8	-- mainharray 
		LDA 0 6	
		CUP 4 init
		...
\end{lstlisting}	
\section*{Esercizio 3}
L'implementazione proposta è stata suddivisa tra i seguenti file:
\begin{description}
	\item[Lexer.l:] esegue, oltre al riconoscimento dei singoli token, il passaggio dei loro parametri posizionali riferiti allo stream di input al Parser,
	\item[Parser.y:] per costruire la struttura dati, in modo coerente anche durante il runtime, in concomitanza alla scansione dei singoli token, utilizza i puntatori alla struttura dati.
		La funzione built-in d'errore \texttt{yyerror} restituisce la posizione raggiunta dal Lexer qualora incontri un errore semantico.
		Nel definire \texttt{yyerror}, in aggiunta alla stringa di argomento con cui viene chiamata, si da la posizione del token a cui l'errore fa riferimento. 
		Nel \texttt{main} è applicato il pretty printing della struttura dati generata dal parser solo nel caso di terminazione con successo del parser.
		\'E stato scelto di implementare il passaggio delle variabili per riferimento in modo da avere una valutazione eager e conseguentemente un inserimento immediato nella struttura dati.
	\item[ABS.c:] definisce la struttura dati e le operazioni necessarie alla sua costruzione e mantenimento.
		La struttura è formata da due sottostrutture:
		\begin{description}
			\item[Command] lista concatenata di assegnazioni di valori alle variabili,
			\item[Section] lista concatenata di sezioni.
		\end{description}
		Con i metodi \texttt{localNameWarning} e \texttt{sectionNameError} si implementa la funzionalità dei messaggi di errore atti a fornire gli elementi necessari all'identificatione delle entità coinvolte e la loro posizione.
		Le funzioni \texttt{sectionSearch} e \texttt{commandValueSearch} eseguono il fetch dei valori delle variabili che sono passate per riferimento.
		La ricerca delle variabili può procedere solo a ritroso; nel caso compaiano riferimenti a variabili non ancora definite viene ritornato un messaggio di errore.
\end{description}

compilato con bison 3.0.4
flex 2.6.0

\end{document}
