%listfiles%prints on standard output the list of files used by this document
\documentclass[a4paper, oneside, 11pt]{article}
%
% useful packages
\usepackage[dvips]{graphicx}
\usepackage[italian]{babel}
\usepackage{amsmath}
\usepackage{amssymb}
% 
%\begin{figure}
%\centering
%\includegraphics[width=\textwidth]{filename.eps}
%\caption{.}
%\label{labelname}
%\end{figure}
%
% to built tables longer than one page (here needed to print all the data)
\usepackage{longtable}
%
% to print numerical values in a format suitable for scientific papers
\usepackage{sistyle}
%\SIstyle{USA}
%\SI{0.01}{m}
%
% fancy footnotes
\usepackage[perpage, symbol*, stable, multiple]{footmisc}
\makeatletter
\renewcommand\@makefnmark{\@textsuperscript{\normalfont (\@thefnmark)}}
\makeatother
%
% fancy captions 
\usepackage[format=default, labelsep=period, font=small, labelfont=sc]{caption}[2004/07/16]
\captionsetup[table]{position=top}
%
% useful general commands
\newcommand*{\eqname}[0]{Eq.}
\newcommand*{\eqspace}[0]{\:}
\providecommand{\eqref}[1]{(\ref{#1})}
%
% redefines the maketitle
\makeatletter
\renewcommand\and{\\}
\renewcommand\maketitle{% 
\bigskip\bigskip\bigskip\bigskip%
\begin{center}\bfseries\large%
Progetto di Linguaggi e Compilatori 1 -- Parte 1 \\ A.A. 2015/16\\%
\end{center}%
\bigskip%
\begin{center}\bfseries\LARGE \@title  \end{center}%
\bigskip%
\begin{center}\bfseries\large \@author \end{center}%
\bigskip\bigskip}
\makeatother
%
% the paths where looking for figures
\graphicspath{{./}, {eps/}}
%
%
\begin{document}
\title{Gruppo 14}
\author{Marco Bucchiarone \and Emanuele Tonetti \and Francesco Zilli}
\maketitle
\section{Esercizio 1}
Il primo esercizio è stato risolto tramite la funzione boundedmaximum , la quale accetta un intero n e una lista di BST x e ritorna una lista di interi, che sono l'elenco di tutti i maxtn degli elementi di ciascun BST. Maxnt è la funzione che passato un BST ritorna l'elemento dello stesso che è il maggiore dei minoranti, che è ricavato applicanto la funzione max, attraverso un foldl1 aumentato di funzionalità, alla lista degli elementi strettamente minori di n ricavata con una funzione di visita del BST in modo da sfruttare la natura della struttura dati considerata. La funzione è testabile attraverso ghci e si forniscono opportuni test case dentro il file es1queryghci.txt, che comprende una rappresentazione più leggibile dei BST proposti in esso.

\section{es2}

\end{document}
